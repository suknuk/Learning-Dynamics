\documentclass[a4paper, 11pt]{article}
\usepackage[utf8]{inputenc}
\usepackage[english]{babel}
\usepackage[utf8]{inputenc}
\usepackage[margin=3.cm]{geometry}
\usepackage{fancyhdr}

\usepackage[T1]{fontenc}
\usepackage{lmodern}

\usepackage{array}

\usepackage{color,array}

\usepackage{float}
\usepackage[table]{xcolor}

\usepackage{multirow}
\usepackage{slashbox} % Defines \backslashbox{..}{..}
\usepackage{adjustbox}

%Includes "References" in the table of contents
\usepackage[nottoc]{tocbibind}

\usepackage{enumitem}
\setlist{nolistsep}

\fancypagestyle{plain}{%
  \renewcommand{\headrulewidth}{0pt}%
  \fancyhf{}%
  \fancyfoot[L]{\footnotesize Raymond Lochner}
  \fancyfoot[C]{\footnotesize October 2017 - Université libre de Bruxelles}%
  \fancyfoot[R]{\thepage}
}
\pagestyle{plain}

\date{\today}

\begin{document}

\begin{titlepage}
	\centering
	{\scshape\LARGE Université libre de Bruxelles \par}
	\vspace{1cm}
	{\scshape\Large INFO-F-409 - Learning Dynamics\par}
	\vspace{1.5cm}
	{\huge\bfseries {Assignment One\par}}
	\vspace{2cm}
	{\Large Raymond Lochner - 000443637\par}
	\vspace{0.5cm}
	{\Large raymond.lochner@ulb.ac.be}
	\vfill
	
	\tableofcontents

\vfill
% Bottom of the page
	{\large \today \par}
\end{titlepage}

\newpage

\section{The Hawk-Dove game}

The Hawk-Dove game was first formulated by John Maynard Smith and Georg Prince in 1973 \cite{MaynardSmith1973}. The aim of the game is to gain a better understanding of conflicts in the animal kingdom. It consits of two players \{Player One, Player Two\} who have each two actions \{Hawk, Dove\}.The resulting payoff matrix can bee seen in Table \ref{tab-HawkDoveOriginal} where:
\begin{itemize}
  \item V = fitness value of winning resources in fight
  \item D = fitness costs of injury
  \item T = fitness costs of wasting time
\end{itemize}

and we assume that V,D,T $\geq$ 0.

\begin{table}[H]
\centering
\caption{Hawk-Dove Payoff Matrix}
\label{tab-HawkDoveOriginal}
\begin{tabular}{ll|l|l|}
\cline{3-4}
                                                  					&      & \multicolumn{2}{c|}{Player Two}                                \\ \cline{3-4}
                                                  					&      & \multicolumn{1}{c|}{Hawk}             & \multicolumn{1}{c|}{Dove}          \\ \cline{1-4}
\multicolumn{1}{|l|}{\multirow{2}{*}{Player One}} & Hawk & \backslashbox[35mm]{(V-D)/2}{(V-D)/2} & \backslashbox[35mm]{V}{0}          \\ \cline{2-4}
\multicolumn{1}{|l|}{}                            					& Dove & \backslashbox[35mm]{0}{V}             & \backslashbox[35mm]{V/2-T}{V/2-T}  \\ \cline{1-4}
\end{tabular}
\end{table}

In a mixed strategy game, we consider each player performing his action with a certain probability \textit{p}, which results in the following payoff matrix displayed in Table \ref{tab-HawkDoveMixedStrategy}.

\begin{table}[H]
\centering
\caption{Hawk-Dove Probability Payoff Matrix}
\label{tab-HawkDoveMixedStrategy}
\begin{tabular}{ll|l|l|}
\cline{3-4}
                                                  					&      & \multicolumn{2}{c|}{Player Two}                                \\ \cline{3-4}
                                                  					&      & \multicolumn{1}{c|}{P(Hawk) = q}             & \multicolumn{1}{c|}{P(Dove) = 1-q}          \\ \cline{1-4}
\multicolumn{1}{|l|}{\multirow{2}{*}{Player One}} & P(Hawk) = p & \backslashbox[35mm]{(V-D)/2}{(V-D)/2} & \backslashbox[35mm]{V}{0}          \\ \cline{2-4}
\multicolumn{1}{|l|}{}                            					& P(Dove) = 1-p & \backslashbox[35mm]{0}{V}             & \backslashbox[35mm]{V/2-T}{V/2-T}  \\ \cline{1-4}
\end{tabular}
\end{table}





\section{Which social dilemma?}

\section{Sequential truel}

%% Bibliography start
\newpage

\bibliographystyle{unsrt}
\bibliography{bibliography}

\end{document}
